\documentclass[a4paper,12pt,leqno,titlepage]{article}
\usepackage{hyperref} %linkitys
\usepackage{graphicx}
\usepackage{listings}
\usepackage{moreverb}
\usepackage{amsmath}
\usepackage{amsthm}
\usepackage[english]{babel}
%\usepackage[finnish]{babel}
\usepackage{ucs}
\usepackage[utf8x]{inputenc}
\usepackage{amssymb}
\newcommand{\R}{\mathbb{R}} %lukujoukkosymbolit
\newcommand{\C}{\mathbb{C}}
\newcommand{\Q}{\mathbb{Q}}
\newcommand{\N}{\mathbb{N}}
\newcommand{\Z}{\mathbb{Z}}
\newcommand{\logM}{\mathcal{M}}
\newcommand{\bigO}{\mathcal{O}}
\setlength{\parindent}{0pt} %kappalejakoa
\setlength{\parskip}{2ex}
\newcommand{\compcent}[1]{\vcenter{\hbox{$#1\circ$}}}
\newcommand{\comp}{\mathbin{\mathchoice
{\compcent\scriptstyle}{\compcent\scriptstyle}
{\compcent\scriptscriptstyle}{\compcent\scriptscriptstyle}}}

\hyphenpenalty=750
\setlength{\emergencystretch}{1.5 em} % Tavutusasetukset suomen kielelle

\hypersetup{
pdfborder = {0 0 0 0}, %linkkien värejä etc kikkailua
colorlinks = true,
linkcolor = black,
urlcolor = blue,
citecolor = red,
}


\usepackage{lastpage} %sivumäärä alaviitteeseen
\usepackage{fancyhdr}

\pagestyle{fancy}
\cfoot{Page \thepage/\pageref{LastPage}} % \\ Opiskelijanumero 013126382}


\begin{document}
\begin{titlepage}
\title{Data structures project, \\
Definitions document}
\author{Heikki Haapala and Aleksi Markkanen\\
Student numbers 014090190 and 013126382\\
\pageref{LastPage} pages}
\date{\today}
\end{titlepage}
\maketitle
\pagebreak
\tableofcontents
\pagebreak




%Määrittelydokumentti
%Mitä algoritmeja ja tietorakenteita toteutat työssäsi
%Mitä ongelmaa ratkaiset ja miksi valitsit kyseiset algoritmit/tietorakenteet
%Mitä syötteitä ohjelma saa ja miten näitä käytetään
%Tavoitteena olevat aika- ja tilavaativuudet (m.m. O-analyysi)
%Lähteet


\section{Introduction}
The topic of choice for the project is the problem of finding the convex hull for a set of points.
This is a problem in the field of computational geometry.
%Valitsin aiheekseni konveksin peitteen ongelman, joka kuuluu laskennallisen geometrian piiriin.
The convex hull of a set of points can be defined thusly.
%Olkoon $S = \{(x_1,y_1),\ldots,(x_n,y_n)\}$ (äärellinen) joukko tason pisteitä.
Let $S = \{(x_1,y_1),\ldots,(x_n,y_n)\}$ be a finite set of points that lie on a plane.
%Tällöin joukon $S$ virittämä \emph{convex hull} on pienin konveksi joukko $C$, jolle $S \subset C$.
Then we say that the convex hull of the set $S$ is the smallest convex set $C$ that satisfies that $S \subset C$.
%Helposti nähdään, että jos $S' \subset S$ niin vastaaville \emph{convex hulleille} $C$ ja $C'$ pätee $C' \subset C$.
It is easy to see that if $S' \subset S$ then for the convex hulls $C$ and $C'$, respectively, it holds that $C' \subset C$.

%Pätee myös, että $$ C = \left\{ \sum_{i=1}^n \alpha_ix_i | x\in S, \sum_{i=1}^n \alpha_i = 1 \right\}. $$

%Koska ongelmanmäärittelyssä $S$ on äärellinen, joukko $C$ on konveksi polygoni.
%Tällöin ongelman ratkaisu on lista $C$:n vertexeistä.
Since $S$ is finite, the set $C$ is a convex polygon.
Thus, we know $C$ uniquely from the set of its vertices.

%Harjoitustyössäni vertailen kahden algoritmin keskinäistä suoriutumista eri tilanteissa.
We will compare the performance of several algorithms in different situations.
\pagebreak
\section{Chosen algorithms and data structures}
We will implement three different algorithms: \emph{Gift-wrapping algorithm}, \emph{Graham scan} and \emph{Quickhull}.
We also implement the \emph{Akl-Toussaint heuristics} to discard large amount of points as a preprosessing step.
%Päätin toteuttaa kaksi eri algoritmia \emph{Akl-Toussaint-heuristiikalla} ja ilman.
%Valitsemani algoritmit ovat \emph{Gift-wrapping algorithm} sekä \emph{Graham scan}.

We will also need a sorting algorithm since the \emph{Graham scan} supposes that the input is ordered.
Points will be stored in a linked list.
%Näiden algoritmien lisäksi tarvitsen myös järjestysalgoritmin, sillä kummatkin edellämainituista algoritmeista olettavat pistejoukon $S$ olevan leksikografisesti järjestetyssä listassa.
%Järjestysalgoritmiksi valitsin \emph{Heapsortin} hyvän aikavaativuuden vuoksi. Käytän siis binäärikekoa tietorakenteenani.

\pagebreak
\section{Inputs}
%Ohjelmalle annetaan syötteenä lista pisteistä, joiden \emph{convex hull} tulee selvittää.
%Syöte järjestetään pisteparin $(x,y)$ luonnolliseen leksikografiseen järjestykseen.
Our program will take as input a list of points for which the convex hull needs to be computed.
%First step is to sort the input.

When using the \emph{Akl-Toussaint heuristics} we will choose from the inputted points those with the largest and smallest $x$ and $y$ coordinates.
We will also pick points that have the minimum and maximum sums and differences of $x$ and $y$ coordinates.
We can safely discard all the points that lie in the thusly formed octagon.
%\emph{Akl-Toussaint-heuristiikkaa} käytettäessä pistejoukosta valitaan $x$- ja $y$-koordinaateiltaan suurimmat ja pienimmät.
%Jos näin saadaan neljä eri pistettä, pisteet muodostavat konveksin nelikulmion, ja valitut neljä pistettä kuuluvat koko pistejoukon \emph{convex hulliin}.
%Tällöin voimme poistaa tarkastelusta pisteet, jotka jäävät näin muodostetun nelikulmion sisäpuolelle.

%Järjestämisen ja esikäsittelyn jälkeen pisteet syötetään algoritmeille.
After the preprosessing we will input the points to the algorithms of choice.

\pagebreak
\section{Computational complexity}

%As the chosen algorithms suppose that the input is sorted, the lower bound for the time complexity is $\bigO(n\log n)$.
%Koska valitsemani algoritmit olettavat syötteen olevan järjestetty, alaraja koko ohjelman aikavaativuudelle on $\bigO(n\log n)$.
%Valitsin \emph{Heapsortin}, joka tunnetusti toimii ajassa $\bigO(n\log n)$.
%We will sort the input using \emph{Mergesort} that has the time complexity of $\bigO(n\log n)$.

%\emph{Gift-wrapping algorithm} on kiinnostava algoritmi, sillä sen aikavaativuus riippuu syötteen lisäksi myös tulosteesta.
%Algoritmin aikavaativuus on $\bigO(hn)$ missä $h$ on vertexien määrä.
%Vertexien määrä voi riippua korkeintaan lineaarisesti syötteen pituudesta, joten algoritmi kuuluu luokkaan $\Theta(n^2)$.
\emph{Gift-wrapping algorithm} is an output sensitive algorithm.
Its time complexity depends on its output.
The time complexity is $\bigO(hn)$ where $h$ is the amount of vertices of the convex hull.
As the amount of vertices can only depend linearly on the input length, the worst performance is $\Theta(n^2)$.

\emph{Graham scan} has the time complexity of $\bigO(n)$ for ordered input.
We will sort the input using \emph{Mergesort} that has the time complexity of $\bigO(n\log n)$.
This brings the total complexity to $\bigO(n\log n)$.

\emph{Quickhull} has the average case complexity of $\bigO(n\log n)$ but it can degenerate to $\bigO(n^2)$ in the worst case.

Space complexity is in every case $\bigO(1)$.


\pagebreak


\begin{thebibliography}{9}
\bibitem{wiki}

Convex hull algorithms,

Wikipedia, the free encyclopedia

\url{http://en.wikipedia.org/wiki/Convex_hull_algorithms}


\end{thebibliography}

\end{document}
