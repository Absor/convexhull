\documentclass[a4paper,12pt,leqno,titlepage]{article}
\usepackage{hyperref} %linkitys
\usepackage{graphicx}
\usepackage{listings}
\usepackage{moreverb}
\usepackage{amsmath}
\usepackage{amsthm}
\usepackage[english]{babel}
%\usepackage[finnish]{babel}
\usepackage{ucs}
\usepackage[utf8x]{inputenc}
\usepackage{amssymb}
\newcommand{\R}{\mathbb{R}} %lukujoukkosymbolit
\newcommand{\C}{\mathbb{C}}
\newcommand{\Q}{\mathbb{Q}}
\newcommand{\N}{\mathbb{N}}
\newcommand{\Z}{\mathbb{Z}}
\newcommand{\logM}{\mathcal{M}}
\newcommand{\bigO}{\mathcal{O}}
\setlength{\parindent}{0pt} %kappalejakoa
\setlength{\parskip}{2ex}
\newcommand{\compcent}[1]{\vcenter{\hbox{$#1\circ$}}}
\newcommand{\comp}{\mathbin{\mathchoice
{\compcent\scriptstyle}{\compcent\scriptstyle}
{\compcent\scriptscriptstyle}{\compcent\scriptscriptstyle}}}

%\hyphenpenalty=750
%\setlength{\emergencystretch}{1.5 em} % Tavutusasetukset suomen kielelle

\hypersetup{
pdfborder = {0 0 0 0}, %linkkien värejä etc kikkailua
colorlinks = true,
linkcolor = black,
urlcolor = blue,
citecolor = red,
}


\usepackage{lastpage} %sivumäärä alaviitteeseen
\usepackage{fancyhdr}

\pagestyle{fancy}
\cfoot{Page \thepage/\pageref{LastPage}} % \\ Opiskelijanumero 013126382}


\begin{document}
\begin{titlepage}
\title{Data structures project, \\
Instructions manual}
\author{Heikki Haapala and Aleksi Markkanen\\
Student numbers 014090190 and 013126382\\
\pageref{LastPage} pages}
\date{\today}
\end{titlepage}
\maketitle
\pagebreak
\tableofcontents
\pagebreak



\begin{comment}
Käyttöohje
Miten ohjelma suoritetaan, miten eri toiminnallisuuksia käytetään
Minkä muotoisia syötteitä ohjelma hyväksyy
Missä kansiossa on jar ja ajamiseen tarvittavat testitiedostot.
\end{comment}

\section{Running the program}
The program requires Java Runtime Environment version 7.

The packaged program \texttt{ConvexHull.jar} can be found in the project subfolder \texttt{jar}.
The program can be run on command line by typing (in the jar folder):

\texttt{java -jar ConvexHull.jar}

The program asks all required arguments which can be typed in the command line.

The program can also be run by adding the arguments after the main command:

\texttt{java -jar ConvexHull.jar arg1 arg2 arg3 ...}

Basic test material can be found in the project subfolder \texttt{testmaterial}.

\pagebreak
\section{Arguments}
The program accepts the arguments in the following order:
\begin{enumerate}
\item input filename

The location of the input file. Accepts points file formatted with one point per line, x-coordinate first followed by whitespace and y-coordinate, for example \emph{Octave} output.

\item \texttt{at} or \texttt{noat}

Chooser for Akl-Toussaint heuristic. \texttt{at} to use and \texttt{noat} to not use the heuristic.

\item integer

The number of iterations to run the main algorithm for (not Akl-Toussaint).

\item \texttt{gift}, \texttt{quick} or \texttt{graham}

Chooser for the main algorithm.
\texttt{gift} to use Gift wrapping algorithm,
\texttt{quick} to use QuickHull algorithm or
\texttt{graham} to use Graham scan algorithm.

\item output filename or \texttt{print}

The location of the output file, will be overwritten or \texttt{print} to print hull points to console.

\item \texttt{yes} or \texttt{no}

\texttt{yes} to draw with black background. \texttt{no} to draw with white background.

\item \texttt{draw} or \texttt{nodraw}

\texttt{draw} to show graphical output of the points. \texttt{nodraw} to skip graphical output.
\end{enumerate}

\pagebreak
%\section{Ohjelman toiminnan empiirisen testauksen tulosten esittäminen graafisessa muodossa}
\section{Example run}
\begin{verbatim}
user@localhost:~/jar$ java -jar ConvexHull.jar
../testmaterial/test100 at 100 quick print yes nodraw

Points read from file: ../testmaterial/test100
Input: a list of 100 points.


Using Akl-Toussaint heuristic.
7
Akl-Toussaint heuristic removed 89 nodes.
Akl-Toussaint heuristic ran in 1ms.

Using QuickHull algorithm.

100 iterations.
Total run time: 11.0 ms.
Average run time: 0.11 ms.
Output: a list of 11 points.

Printing hull points to console (x y).
-1.187863496743761 2.295679969813418
-0.4306985997200181 2.141215375336973
0.6889983525066655 1.605248823316989
1.856457253218937 0.8404406772993155
2.050621638328523 0.508194457316787
2.638621913585121 -1.575950443795056
0.4117650626370341 -2.869145785655391
-1.000108998272335 -1.972846564433562
-1.612219329446485 -1.527331902358966
-2.014959422427117 -1.226536673500165
-2.278175791373962 1.894395038709398

Drawing with awesome colours!

Not drawing points on screen.
\end{verbatim}

\end{document}
